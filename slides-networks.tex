\documentclass[professionalfonts,12pt]{beamer}

\usepackage{minted}

\newcommand\paper[1]{%
  \begin{textblock*}{\paperwidth}(2pt,0.98\textheight)
    \raggedleft{\tiny #1}\hspace{.5em}
  \end{textblock*}}

\usepackage[utf8]{inputenc}
\usepackage[T1]{fontenc}
\usepackage[protrusion=true,expansion=true]{microtype}
\usepackage[normalem]{ulem} 
\usepackage{amsmath}

%\usepackage[loosequotes,slides]{MinionPro}

\usepackage{bera,berasans,beramono}

\usepackage{tikz}

%%%%%%%%% COLORS

\definecolor{clouds}{HTML}{ECF0F1}

\definecolor{belize}{HTML}{2980B9}
\definecolor{peter}{HTML}{3498DB}

\definecolor{pomegranate}{HTML}{C0392B}

\definecolor{turquoise}{HTML}{1ABC9C}

\definecolor{midnight}{HTML}{2C3E50}

%\setbeamertemplate{background canvas}[vertical shading][top=clouds, bottom=clouds]

\makeatletter
\setbeamertemplate{background canvas}{%
   \ifnum\c@framenumber=1%
      \color{turquoise}\rule{\paperwidth}{\paperheight}
   \fi%
}
\makeatother

\setbeamercolor{normal text}{fg = midnight}
\setbeamercolor{alerted text}{fg = pomegranate}

\setbeamercolor{frametitle}{fg = belize}
\setbeamercolor{framesubtitle}{fg = peter}

\setbeamercolor{title}{fg = white}

\setbeamercolor{structure}{fg = turquoise}

%%%%%%%% FONTS

\setbeamerfont{title}{series=\sffamily}
\setbeamerfont{subtitle}{series=\sffamily}
\setbeamerfont{frametitle}{series=\sffamily}
\setbeamerfont{framesubtitle}{series=\sffamily}


\title{A (not so) gentle introduction to ecological networks}
\author{Timothée Poisot}
\institute{Université du Québec à Rimouski}
\date{\today}

\begin{document}

\frame[plain]{\titlepage}

\section{Introduction}

\begin{frame}{Why should I care about networks?}
   \begin{itemize}
      \item A good way to harness complexity
      \item A solid mathematical foundation
      \item Elegant algorithms
      \item That guy is going to talk about them for, like, two hours\ldots
   \end{itemize}
\end{frame}

\begin{frame}{What is a network?}{A mathematical approach}
   A \emph{graph} is a \textbf{representation} of a \textbf{set of objects} where some pairs of objects are \textbf{connected} by \textbf{links}.
   \vskip 2em
   Or more formally, $G = (V,E)$, a \emph{graph} $G$ is an ordered pair of \emph{vertices} $V$ linked together by \emph{edges} $E$.
   \vskip 2em
   Each element of $E$ is a two-element subset of $V$.
   \vskip 2em
   The \emph{order} of a graph is $|V|$, and its \emph{size} is $|E|$.
\end{frame}

\begin{frame}{What is a network?}{An example}
   In an omnivory scenario, one top predator $P$ consumes both an intermediate consumer $C$ and a primary producer $R$. The intermediate consumer also consumes the producer.
   \vskip 2em
   This network is specified by
   $$G = \left(\{P, C, R\},\{\{P,C\},\{P,R\},\{C,R\}\}\right) $$
   \vskip 2em
   Or for brevity
   $$G = \left(\{P, C, R\},\{PC,PR,CR\}\right) $$
\end{frame}

\begin{frame}{What is a network?}{The adjacency matrix}
   Networks are often represented by their \textbf{adjacency matrix}.
   \vskip 2em
   The adjacency matrix $\mathbf{A}$ of a graph $G = (V,E)$ has elements $A_{ij}$ with value 1 if there is an edge between the node $V_i$ and the node $V_j$, and 0 otherwise.
   \vskip 2em
   It follows that $\sum_i\sum_j A_{ij} = |E|$.
\end{frame}

\begin{frame}{What is a network?}{Edge direction}
   Edges can be \emph{directed} (arcs, directed edges) or not. An edge between a vertex and itself (cannibalism) is a \emph{self-loop}.
   \vskip 2em
   In an \textbf{undirected graph}, there are at most $|V|(|V|-1)/2$ edges if there are no \emph{self-loops}.
   \vskip 2em
   In a \textbf{directed graph}, there are at most $|V|(|V|-1)$ edges if there are no \emph{self-loops}.
   \vskip 2em
   \alert{Exercice}: What is the maximal size of a graph of order $n$ if there are self-loops?
\end{frame}

\begin{frame}{What is a network?}{Edge weight}
   Edges in a network can have a \textbf{weight} (for example, the number of contacts between individuals).
   \vskip 2em
   The elements of the adjacency matrix $\mathbf{A}$ can be given \emph{continuous} values.
   \vskip 2em
   It's possible to work both on the \emph{weighted} and \emph{unweighted} properties of a graph. \alert{However}, there are many methods that (as of now) can only be applied to \textbf{undirected, unweighted} networks.
\end{frame}

\begin{frame}{Number of partners}
   The number of vertices \emph{receiving} a link from a focal vertex are called its \textbf{successors}
   \vskip 2em
   The number of vertices \emph{establishing} a link towards a focal vertex are called its \textbf{predecessors}
   \vskip 2em
   The \emph{total} number of edges connected to a focal vertex is this vertex \textbf{degree}
\end{frame}

\begin{frame}{Where is the ecology in all that?}
   \begin{description}
      \item[graph] The whole community, \emph{i.e.} the populations and their interactions
      \item[vertices] The composition of the community (species present)
      \item[edges] The interactions between the populations
   \end{description}
\end{frame}

\begin{frame}{Where is the ecology in all that?}{Exemple of ``networkable'' systems}
   \begin{itemize}
      \item Trophic systems
      \item Plant--pollinators
      \item Hosts--parasites
      \item Mutualism
      \item Social interactions
   \end{itemize}
   \vskip 2em
   Any system in which the \textbf{same ecological interaction} happens several time in a community can (should) be studied using network theory
\end{frame}

\section{Representing networks}

\section{Network-level properties}

\begin{frame}{Connectance}
   Connectance is the \emph{proportion of possible interactions realized}
   \vskip 2em
   The order of an ecological network is usually called $S$, and its size $L$
   \vskip 2em
   In a network with directed edges and self-loops, the connectance is $Co = L/S^2$
   \vskip 2em
   \alert{Exercice}: What is the expression of the connectance for directed/undirected networks with/without self-edges?
\end{frame}

\section{Vertex-level properties}

\begin{frame}{Number of partners}
   The number of (\emph{e.g.}) preys of a predator is its \textbf{generality} (number of successors)
   \vskip 2em
   The number of (\emph{e.g.}) predators of a prey is its \textbf{vulnerability} (number of predecessors)
\end{frame}

\begin{frame}[fragile]{Number of partners}
   \begin{minted}[linenos]{splus}
      web = read.table('web.dat')
      generality = rowSums(web)
      vulnerability = colSums(web)
      degree = generality + vulnerability
   \end{minted}
\end{frame}

\section{Generating networks}

\begin{frame}{The niche model of food webs}

\end{frame}

\end{document}
